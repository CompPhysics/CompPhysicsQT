%%
%% Automatically generated file from DocOnce source
%% (https://github.com/doconce/doconce/)
%% doconce format latex program.do.txt --minted_latex_style=trac --latex_admon=paragraph --no_mako
%%


%-------------------- begin preamble ----------------------

\documentclass[%
oneside,                 % oneside: electronic viewing, twoside: printing
final,                   % draft: marks overfull hboxes, figures with paths
10pt]{article}

\listfiles               %  print all files needed to compile this document

\usepackage{relsize,makeidx,color,setspace,amsmath,amsfonts,amssymb}
\usepackage[table]{xcolor}
\usepackage{bm,ltablex,microtype}

\usepackage[pdftex]{graphicx}

\usepackage[T1]{fontenc}
%\usepackage[latin1]{inputenc}
\usepackage{ucs}
\usepackage[utf8x]{inputenc}

\usepackage{lmodern}         % Latin Modern fonts derived from Computer Modern

% Hyperlinks in PDF:
\definecolor{linkcolor}{rgb}{0,0,0.4}
\usepackage{hyperref}
\hypersetup{
    breaklinks=true,
    colorlinks=true,
    linkcolor=linkcolor,
    urlcolor=linkcolor,
    citecolor=black,
    filecolor=black,
    %filecolor=blue,
    pdfmenubar=true,
    pdftoolbar=true,
    bookmarksdepth=3   % Uncomment (and tweak) for PDF bookmarks with more levels than the TOC
    }
%\hyperbaseurl{}   % hyperlinks are relative to this root

\setcounter{tocdepth}{2}  % levels in table of contents

\usepackage[framemethod=TikZ]{mdframed}

% --- begin definitions of admonition environments ---

% --- end of definitions of admonition environments ---

% prevent orhpans and widows
\clubpenalty = 10000
\widowpenalty = 10000

% --- end of standard preamble for documents ---


% insert custom LaTeX commands...

\raggedbottom
\makeindex
\usepackage[totoc]{idxlayout}   % for index in the toc
\usepackage[nottoc]{tocbibind}  % for references/bibliography in the toc

%-------------------- end preamble ----------------------

\begin{document}

% matching end for #ifdef PREAMBLE

\newcommand{\exercisesection}[1]{\subsection*{#1}}


% ------------------- main content ----------------------



% ----------------- title -------------------------

\thispagestyle{empty}

\begin{center}
{\LARGE\bf
\begin{spacing}{1.25}
Including Quantum Technologies and AI/ML in educational programs at the Department of Physics, UiO
\end{spacing}
}
\end{center}

% ----------------- author(s) -------------------------

\begin{center}
{\bf Marianne Etzelm\"uller Bathen, Morten Hjorth-Jensen, and Lasse Vines}
\end{center}

    \begin{center}
% List of all institutions:
\centerline{{\small Department of Physics, UiO}}
\end{center}
    
% ----------------- end author(s) -------------------------


% --- begin date ---
\begin{center}
Planned start: Fall 2024 for new study direction (first point below)
\end{center}
% --- end date ---

\vspace{1cm}


% !split
\subsection*{Establishing new study directions in the Physics and Astronomy BSc program and Master of Science in Physics}


% --- begin paragraph admon ---
\paragraph{}
We would like to propose
\begin{enumerate}
\item A new study direction under the Physics and Astronomy (PA) BSc program called
\begin{itemize}

  \item \textbf{Quantum technologies and AI/ML} (name to be discussed)

  \item Planned start fall 2024 for the new study direction

\end{itemize}

\noindent
\item At a later stage, a possible name change of the PA BSc program to for example
\begin{itemize}

  \item Physics, Astronomy and Quantum Technologies

\end{itemize}

\noindent
\item Similarly, the Physics MSc program changes name to
\begin{itemize}

  \item Physics and Quantum Technologies

  \item With a study direction in Quantum Technologies/Science

  \item and a study direction Computational Physics and AI/ML for the Physical Sciences
\end{itemize}

\noindent
\end{enumerate}

\noindent
% --- end paragraph admon ---




% --- begin paragraph admon ---
\paragraph{Possible collaboration with:}
\begin{itemize}
 \item Department of Chemistry

 \item Department of Informatics

 \item Department of Mathematics
\end{itemize}

\noindent
% --- end paragraph admon ---



% !split
\subsection*{Strategic importance}


% --- begin paragraph admon ---
\paragraph{}
Computational physics, computational science and data science play a
central role in scientific investigations and are central to
innovation in most domains of our lives. These fields underpin the
majority of today's technological, economic and societal feats. We
have entered an era in which huge amounts of data offer enormous
opportunities, but only to those who are able to harness them. The 3rd
industrial revolution will alter significantly the demands on the
workforce. In particular, the developments taking place in quantum
technologies and quantum information systems (QIS) together with
artificial intelligence (AI) and machine learning (ML) are expected to
play a significant role in technology developments and innovations,
and for fundamental discoveries in physics.
% --- end paragraph admon ---



% !split
\subsection*{AI and machine learning}


% --- begin paragraph admon ---
\paragraph{}
Artificial
intelligence is built upon integrated machine learning algorithms,
which in turn are fundamentally rooted in optimization and statistical
learning.
% --- end paragraph admon ---



% !split
\subsection*{AI and ML in Physics}


% --- begin paragraph admon ---
\paragraph{}
Artificial intelligence (AI) and Machine learning (ML) techniques have
in the last years gained considerable traction in scientific
discovery. In particular, applications and techniques for so-called
\textbf{fast ML}, that is high-performance ML methods applied to real time
experimental data processing, hold great promise for enhancing
scientific discoveries in many different disciplines.  These
developments cover a broad mix of rapidly evolving fields, from the
development of ML techniques to computer and hardware architectures.
% --- end paragraph admon ---



% !split
\subsection*{Physics based Machine Learning}

An important and emerging field is what has been dubbed as scientific ML, see the article by Deiana et al \href{{https://arxiv.org/abs/2110.13041}}{Applications and Techniques for Fast Machine Learning in Science, arXiv:2110.13041}


% --- begin paragraph admon ---
\paragraph{}
The authors discuss applications and techniques for fast machine
learning (ML) in science -- the concept of integrating power ML
methods into the real-time experimental data processing loop to
accelerate scientific discovery. The report covers three main areas

\begin{enumerate}
\item applications for fast ML across a number of scientific domains;

\item techniques for training and implementing performant and resource-efficient ML algorithms;

\item and computing architectures, platforms, and technologies for deploying these algorithms.
\end{enumerate}

\noindent
% --- end paragraph admon ---



% !split
\subsection*{Many new research directions}


% --- begin paragraph admon ---
\paragraph{}
For our research in for example particle and nuclear physics, fields
which cover a huge range of energy and length scales, spanning from
our smallest constituents to the physics of dense astronomical objects
like supernovae and neutron stars, AI and ML techniques offer
possibilities for new discoveries and deeper insights about the
physics of atomic nuclei, elementary particles and dense
matter. Similarly, ML algorithms are widely applied in condensed
matter physics, materials science and nanotechnology, in molecular
dynamics simulations of complex systems in neuroscience and in many
other fields in natural science.
% --- end paragraph admon ---




% --- begin paragraph admon ---
\paragraph{Examples of applications in subatomic physics.}
\begin{itemize}
\item \textbf{Artificial Intelligence and Machine Learning in Nuclear Physics}, Amber Boehnlein et al., \href{{https://arxiv.org/abs/2112.02309}}{arXiv:2112.02309} and Reviews of Modern Physics, 2022, in press

\item \textbf{Predicting Solid State Material Platforms for Quantum Technologies}, Hebnes et al,. \href{{https://arxiv.org/abs/2203.16203}}{arXiv:2203.16203}

\item \href{{https://arxiv.org/abs/1803.08823}}{Mehta et al.} and \href{{https://www.sciencedirect.com/science/article/pii/S0370157319300766?via%3Dihub}}{Physics Reports (2019)}.

\item \href{{https://link.aps.org/doi/10.1103/RevModPhys.91.045002}}{Machine Learning and the Physical Sciences by Carleo et al}

\item \href{{https://pdg.lbl.gov/2021/reviews/rpp2021-rev-machine-learning.pdf}}{Particle Data Group summary on ML methods}
\end{itemize}

\noindent
% --- end paragraph admon ---



% !split
\subsection*{Quantum Information Technologies (QIT)}


% --- begin paragraph admon ---
\paragraph{}
Recent developments in quantum information systems
and technologies offer the possibility to address some of the most
challenging large-scale problems, whether they are represented by
complicated interacting quantum mechanical systems or classical
systems.  Originally proposed by Feynman, the efficient simulation of
for example quantum systems by other, more controllable quantum
systems formed the basis for modern constructions of quantum
computations.  Many algorithmic and theoretical advances have followed
since the initial work in this area and with recent developments in
quantum computing hardware there is an additional drive to identify
early practical problems on which these devices might demonstrate an
advantage.
% --- end paragraph admon ---



% !split
\subsection*{More on QIT}


% --- begin paragraph admon ---
\paragraph{}
In addition to theoretical activities conducted at the Department of
Physics (mainly at the Center for Computing in Science Education
(CCSE) and the condensed matter group and other groups), there is a
growing interest to study candidate systems for making quantum
hardware. In particular, so-called point defects in semiconductors are
pursued by experimenters at the center for Materials Science.  With
this broad list of activities at the department of physics, there is a
huge potential to prepare the ground for educating physicists with the
theoretical and experimental background needed for the 21st
century. There is also a great interest in candidates with such a
background, knowledge, skills and competences in industry and the
public sector.
% --- end paragraph admon ---



% !split
\subsection*{Why such a change?}


% --- begin paragraph admon ---
\paragraph{}
Establishing such educational directions will be unique in Norway and
has the potential to attract excellent students.  The popularity of
the Computational Science and in particular the Computational Physics
and Computational Materials Science study direction are clear
indicators that these are fields with the potential to attract new
students.
% --- end paragraph admon ---




% --- begin paragraph admon ---
\paragraph{}
Oslo Metropolitan university has recently acquired two quantum
quantum computers and is now establishing research and educational
initiatives in quantum information systems. There are thus several
interesting avenues for joint collaborations in quantum information
systems and quantum technologies as well as developing joint
educational programs.
% --- end paragraph admon ---



% !split
\subsection*{More on motivation}

Computational physics plays a central role in the above mentioned
developments.  Computations are simply indispensable.  At the
department of physics of the university of Oslo this is reflected in
the extremely popular study direction Computational Physics of the
master of science (MSc) program Computational Science. This program
has over the last two decades recruited many excellent students,
resulting in highly attractive candidates in academia and in industry
and the public sector. A large fraction of these students have
specialized either in artificial intelligence and machine learning
and/or in quantum information systems.  The large majority of the
these students have job offers at least one year before completing
their MSc theses. The program has also become one of the most
selective master programs at the University of Oslo, requiring a grade
average of 4.7 for entry in 2021. Furthermore, with recent advances in
quantum technologies, there is a strong potential for new developments
in the fields of nanotechnology and materials science, with the
possibilities to develop new experimental activities.

% !split
\subsection*{Rationale}


% --- begin paragraph admon ---
\paragraph{}
The rationale behind proposing such new study directions is:
\begin{enumerate}
\item To attract at an earlier stage new students with an explicit interest in QIS, QT and AI and ML in physics. 

\item To enhance the recruitment to fields in physics which are in high demand for students and candidates with an expertise in computations, QIS, QT, AI and ML. We expect high demands from both the private sector and the public sector for candidates with these competences, insights and skills.

\item Candidates with such a background will be of great importance for new scientific discoveries and technological innovations. At the department of physics of the university of Oslo there are several research directions whose scientific activities will benefit at large from candidates with such a background, spanning from fast ML for new discoveries to the development of QTs.   
\end{enumerate}

\noindent
% --- end paragraph admon ---



% !split
\subsection*{Societal needs}


% --- begin paragraph admon ---
\paragraph{}
The new study direction aims at addressing future societal needs, such as the  needs for specialized candidates (Master of Science, PhDs, postdocs), but also the needs of  people with a broad overview of what is possible in  QIS and QT. There are  not enough potential employees in AI, ML, QIS and QT. There is  a clear supply gap.
% --- end paragraph admon ---




% --- begin paragraph admon ---
\paragraph{}
A BSc degree  with specialization  is thus a good place to start. Linking this with the Physics MSc  program and the Computational Science program and the study directions Computational physics and Computational materials science, will offer to our various research fields top candidates as well as pointing to new research directions.
% --- end paragraph admon ---



% !split
\subsection*{Paths in the BSc program}


% --- begin paragraph admon ---
\paragraph{}
The study direction we propose is
\begin{itemize}
\item \textbf{Quantum technologies and AI/ML in the physical sciences} (name to be discussed)
\end{itemize}

\noindent
% --- end paragraph admon ---



% !split
\subsection*{Structure and courses}

% --- begin paragraph admon ---
\paragraph{}
There are several existing courses which can be included in this
program. There are also courses which need to be established.
We would like to propose three new courses (see tentative course contents below) for the new BSc study direction.
\begin{enumerate}
\item FYS1xxx Introduction to Quantum Technologies, third semester

\item FYS2xxx Quantum Materials, fifth semester

\item FYS3xxx Quantum Computing, sixth semester
\end{enumerate}

\noindent
The first year is identical with the BSc program \textbf{Physics and Astronomy}.
% --- end paragraph admon ---



% !split
\subsection*{Structure and courses}

The table here is an example of a suggested path for a study direction
in quantum technologies and computational physics and AI/ML.


\begin{quote}
\begin{tabular}{llll}
\hline
\multicolumn{1}{l}{  } & \multicolumn{1}{l}{ 10 ECTS } & \multicolumn{1}{l}{ 10 ECTS } & \multicolumn{1}{l}{ 10 ECTS } \\
\hline
6th semester & Elective/ExPhil & Elective/ML courses & FYS3XXX Quantum Computing, new                    \\
\hline
5th semester & FYS2160         & FYS3110             & FYS3XXX Quantum Materials, new                    \\
\hline
4th semester & FYS2130         & FYS2140             & FYS3150/FYS2150                                   \\
\hline
3rd semester & MAT1120         & FYS1120             & FYS1XXX Introduction to Quantum Technologies, new \\
\hline
2nd semester & MAT1110         & STK-FYS1100         & FYS1105                                           \\
\hline
1st semester & MAT1100         & IN1900              & FYS1100                                           \\
\hline
\end{tabular}
\end{quote}

\noindent
% !split
\subsection*{Description of new courses for BSc study direction}

\paragraph{First course: Introduction to quantum technologies, third semester, 10 ECTS.}

% --- begin paragraph admon ---
\paragraph{Content:}
\begin{enumerate}
\item Motivasjon 

\item Basic quantum physics/ QT at a glance 

\item Quantum bits versus classical bits

\item Materials and actual realizations/quantum platforms

\item Quantum sensors

\item Quantum communication and quantum cryptography

\item Quantum computing
\end{enumerate}

\noindent
% --- end paragraph admon ---




% --- begin paragraph admon ---
\paragraph{Learning goals.}
Main objectives: general introduction to quantum technology that provides an overview of the entire field
Understand the difference between qubits and classical bits.
% --- end paragraph admon ---



% !split
\paragraph{Second course: Quantum materials, fifth semester, 10 ECTS.}

% --- begin paragraph admon ---
\paragraph{First part: Condensed matter physics.}
\begin{enumerate}
\item Introduction 

\item Crystal bonding

\item Lattices  

\item Reciprocal space

\item Crystals 

\item Bragg diffraction 

\item Brillouin zones 

\item XRD/TEM lab 

\item Phonons 

\item Vibration in atomic chains 

\item Dispersion relation 

\item Periodic boundary conditions

\item Phonons and heat 

\item Free electron gas 

\item Transport properties of electrons 

\item Electrons in the solid state  

\item Origin of band gap 

\item Bloch functions 

\item Kronig penny model

\item Effective mass model  
\end{enumerate}

\noindent
% --- end paragraph admon ---

 


% --- begin paragraph admon ---
\paragraph{Second part: Quantum materials.}
\begin{enumerate}
\item Trapped ions 

\item Manipulating single atoms 

\item Applications for QT, memory and computing maybe

\item BCS theory

\item Meissner effect and energy gap 

\item Type 1 og type 2 Superconductors 

\item Josephson junctions 

\item SQUID 

\item Quantum dots and point defects

\item Magnetic field sensing 

\item Quantum computing

\item Construction of a quantum computer 
\end{enumerate}

\noindent
% --- end paragraph admon ---




% --- begin paragraph admon ---
\paragraph{Learning goals.}
\textbf{TBA}
% --- end paragraph admon ---



% !split
\paragraph{Third course: Quantum Computing, sixth semester, 10 ECTS.}

% --- begin paragraph admon ---
\paragraph{Content.}
\begin{enumerate}
\item Tensor products of Hilbert Spaces and definitions of Computational basis sets

\item Simple Hamiltonians and other operators

\item Unitary transformations, gates and quantum circuits

\item States and Observables for Composite systems

\item Quantum operations

\item Spectral decomposition and measurements

\item Density matrices

\item Schmidt decomposition

\item Entanglement and Entropy

\item Quantum State Preparation
\begin{itemize}

  \item Single qubit state preparation

  \item Two-qubit state preparation

  \item Two-qubit gate preparation

  \item Four qubit state preparation

\end{itemize}

\noindent
\item Quantum gates and operations

\item Central quantum algorithms

\item Quantum Fourier Transforms

\item Quantum Phase estimation algorithm

\item VQE, Variational Quantum Eigensolver

\item Simulating Hamiltonians on NISQ quantum computers

\item Jordan-Wigner transformations

\item Suzuki-Trotter approximation

\item Running computations on IBM's machines with Qiskit

\item Quantum Partition Function

\item Quantum Tomography

\item Quantum Error Correction
\end{enumerate}

\noindent
% --- end paragraph admon ---




% --- begin paragraph admon ---
\paragraph{Learning goals.}
After completing this course, you are able to: 
\begin{enumerate}
\item apply quantum computing algorithms to selected quantum-mechanical many-particle systems.  

\item describe the differences between quantum and classical computation of quantum mechanical many-particle systems.  

\item discern potential performance gains of quantum vs.~classical algorithms.

\item implement and design quantum circuits for studies of quantum mechanical systems.  

\item run these algorithms on existing quantum computers.  

\item understand the role of noise in quantum computing.  
\end{enumerate}

\noindent
% --- end paragraph admon ---



% ------------------- end of main content ---------------

\end{document}

