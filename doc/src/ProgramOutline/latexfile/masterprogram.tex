\documentclass{beamer}
\usepackage[utf8]{inputenc}
\usepackage{amsmath, amssymb, bm}
\usepackage{physics}
\usepackage{graphicx}
\usepackage{hyperref}
\usepackage{xmpmulti}
\usepackage{tikz}
\usepackage{pgfplots}
\pgfplotsset{compat=newest}
\usepackage{siunitx}
\usepackage{longtable}
\sisetup{round-mode=places,round-precision=1}
\usepackage{braket}
\usetikzlibrary{arrows.meta, shapes.misc, positioning, backgrounds}
\usetikzlibrary{calc, decorations.markings,decorations.pathmorphing}
\usetheme{Madrid} % You can change the theme as you like
\usecolortheme{seagull}





\begin{document}
\title{Master and bachelor programs in quantum technologies}
\author{Morten Hjorth-Jensen}
\institute{Department of Physics and Center for Computing in Science Education, University of Oslo, Norway}
\date{October 27, 2025}


\begin{frame}[plain,fragile]
\titlepage
\end{frame}

\begin{frame}[plain,fragile]
\frametitle{Possible paths}

\begin{block}{Alternative 1 }

\begin{enumerate}
\item Keep present study direction under the Physics and Astronomy (PA) BSc program called
\begin{itemize}

  \item \textbf{Quantum technology}, but increase number of students

\end{itemize}

\noindent
\item Similarly, keep existing study direction under the Physics master of Science program
\begin{itemize}

  \item \textbf{Quantum Science and Quantum Technology}, but increase number of students

\end{itemize}

\noindent
\item And do the same for the CS program with the study direction
\begin{itemize}

  \item \textbf{Quantum Information Science and Technology}, but increase number of students
\end{itemize}

\noindent
\end{enumerate}

\noindent
\end{block}
\end{frame}

\begin{frame}[plain,fragile]
\frametitle{Second alternative}

\begin{block}{Alternative 2 }
\begin{enumerate}
\item Keep present study direction under the Physics and Astronomy (PA) BSc program called
\begin{itemize}

  \item \textbf{Quantum technology}, but increase number of students

\end{itemize}

\noindent
\item Create a new master of science program called
\begin{itemize}

  \item \textbf{Quantum Science and Technology} (or something similar)
\end{itemize}

\noindent
\end{enumerate}

\noindent
This will have as a consequence that one may eventually drop the study directions in the CS and Physics MSc programs (see below).
\end{block}
\end{frame}

\begin{frame}[plain,fragile]
\frametitle{Third  alternative}

\begin{block}{Alternative 3 }
\begin{enumerate}
\item New bachelor program 
\begin{itemize}

  \item \textbf{Quantum Science and Technology} (or something similar)

\end{itemize}

\noindent
\item Create a new master of science program called
\begin{itemize}

  \item \textbf{Quantum Science and Technology} (or something similar)
\end{itemize}

\noindent
\end{enumerate}

\noindent
This will have as a consequence that one may eventually drop the study directions in the CS and Physics MSc programs and Physics Bachelor program.
\end{block}
\end{frame}

\begin{frame}[plain,fragile]
\frametitle{Alternative 2: New master of science program}

\textbf{Quantum Science and Technology}, as a multidisciplinary program much along the same lines as the CS program
\begin{block}{Collaboration with four possible departments  }
\begin{itemize}
 \item Department of Chemistry (quantum chemistry and Hyllerås center, Simen Kvaal and Thomas B. Pedersen, plus several PDs and PhDs)

 \item Department of Informatics (cryptography, cyber security, Audun Jøsang and Mavroeidis, Vasileios, plus PDs and PhDs)

 \item Department of Mathematics (quantum information theory, operator algebra group, Alexander Muller-Hermes, Nadia S. Larsen, Eric Bedos and Sergiy Neshveyev plus several PhDs and PDs)

 \item Department of Physics (theory and experiment, several professors, PDs and PhDs, largest activity )
\end{itemize}

\noindent
\end{block}
The program is hosted by the department of Physics.
\end{frame}

\begin{frame}[plain,fragile]
\frametitle{Core mission of the program}

The master program provides advanced knowledge to develop
cutting-edge theoretical and experimental research in quantum
simulation, quantum computing, quantum sensors, quantum communications
and purely formal theoretical aspects.

Various universities, research centres and companies may actively
participate in teaching the master's degree.
\end{frame}


\begin{frame}[plain,fragile]

\frametitle{Motivation and Vision}

Quantum technologies represent a transformative frontier across
computing, sensing, communication, and materials science. The proposed
MSc program addresses the growing demand for researchers and
professionals capable of integrating quantum physics, mathematics,
informatics, and quantum chemistry into coherent technological and
scientific applications.
\end{frame}

\begin{frame}[plain,fragile]
\frametitle{Potential partners}
The program will be anchored in foundational research and experimental efforts in the Oslo region, leveraging existing expertise at:
\begin{itemize}
    \item University of Oslo 
    \item Oslo Metropolitan University
    \item University of South-Eastern Norway
    \item Simula Research Laboratory  (quantum algorithms and hybrid computing),
    \item SINTEF (quantum devices, sensors, and materials).
\end{itemize}


\end{frame}


\begin{frame}[plain,fragile]
\frametitle{Program Objectives}
The MSc in Quantum Science and Technology will:
\begin{enumerate}
    \item Provide students with a deep understanding of the theoretical, mathematical, and computational foundations of quantum mechanics and information.
    \item Train students in both experimental and computational methods applicable to quantum technologies.
    \item Encourage interdisciplinary collaboration bridging physics, informatics, mathematics, and chemistry.
    \item Prepare graduates for both academic research careers and emerging industrial roles in quantum computing, sensing, communication, and materials.
\end{enumerate}

\end{frame}


\begin{frame}[plain,fragile]
\frametitle{Program Structure}
The program consists of:
\begin{itemize}
\item A \textbf{Basic core curriculum} (60 ECTS) establishing the interdisciplinary foundation.
  \begin{itemize}
    \item A \textbf{specialization track} (30 ECTS) chosen from the study directions listed below.
    \item A \textbf{Master’s thesis} (30 ECTS) supervised by one or more partner institutions.
    \item Or a  \textbf{Master’s thesis} (60 ECTS) supervised by one or more partner institutions.      \end{itemize}
\item Four or more study directions, see below. The study directions, as in the CS program, may reside with a given department.
\end{itemize}

\end{frame}



\begin{frame}[plain,fragile]
\frametitle{Study Direction 1: Quantum Computing and Algorithms}
\textbf{Focus:} Foundational and practical aspects of quantum computation, from physical qubits to high-level algorithms.

\textbf{Content:} Quantum gates, circuits, error correction, variational quantum algorithms (VQE, QAOA), quantum Fourier transform, and quantum machine learning.

\textbf{Disciplinary integration:}
\begin{itemize}
    \item Physics: Quantum mechanics and hardware models.
    \item Mathematics: Linear algebra and optimization.
    \item Informatics: Quantum programming and algorithmic complexity.
    \item Quantum Chemistry: Electronic structure simulations using quantum algorithms.
\end{itemize}

\textbf{Local relevance:} Collaborations with Simula Research Laboratory, SINTEF and OsloMet

\textbf{Career paths:} Researcher, quantum software developer, academic or industrial scientist.

\end{frame}


\begin{frame}[plain,fragile]
\frametitle{Study Direction 2: Quantum Materials and Quantum Simulations}
\textbf{Focus:} Theoretical and computational study of materials with quantum properties.

\textbf{Content:} Many-body theory, DFT, ab-initio methods, tensor networks, and quantum Monte Carlo.

\textbf{Disciplinary integration:}
\begin{itemize}
    \item Physics: Condensed matter and many-body systems.
    \item Mathematics: Numerical methods.
    \item Informatics: Simulation algorithms.
    \item Quantum Chemistry: Correlation and electronic structure.
\end{itemize}

\textbf{Local relevance:} SMN and Hylleraas Centre for Quantum Molecular Sciences.

\textbf{Career paths:} Computational physicist, materials modeler, simulation scientist.

\end{frame}


\begin{frame}[plain,fragile]
\frametitle{Study Direction 3: Quantum Information and Communication}
\textbf{Focus:} Theoretical and experimental foundations of quantum information science and secure communication.

\textbf{Content:} Entanglement, teleportation, quantum Shannon theory, quantum networks, and repeaters.

\textbf{Disciplinary integration:}
\begin{itemize}
    \item Physics: Photonic and spin-based quantum systems.
    \item Mathematics: Information theory and group theory.
    \item Informatics: Communication protocols and cryptographic architectures.
    \item Quantum Chemistry: Photon–matter interactions.
\end{itemize}

\textbf{Local relevance:} UiO, SINTEF and  Simula. OsloMet? USN?

\textbf{Career paths:} Quantum cryptography specialist, communication researcher, network engineer.

\end{frame}


\begin{frame}[plain,fragile]
\frametitle{Study Direction 4: Quantum Sensing and Metrology}
\textbf{Focus:} Quantum-enhanced precision measurement and detection technologies.

\textbf{Content:} Quantum Cramér–Rao bound, entangled sensing, NV centers, superconducting sensors, quantum control.

\textbf{Disciplinary integration:}
\begin{itemize}
    \item Physics: Quantum optics and solid-state physics.
    \item Mathematics: Estimation theory and signal processing.
    \item Informatics: Bayesian inference and data analysis.
    \item Quantum Chemistry: Defects and spin–orbit coupling.
\end{itemize}

\textbf{Local relevance:} SMN  at UiO and SINTEF sensor technologies, USN, Justervesenet, FFI.

\textbf{Career paths:} Quantum instrumentation engineer, metrology researcher, applied physicist.

\end{frame}


\begin{frame}[plain,fragile]
\frametitle{Study Direction 5: Quantum Control and Quantum Engineering}
\textbf{Focus:} Quantum device design, control, and scalability.

\textbf{Content:} Control theory, decoherence, noise modeling, hardware platforms, cryogenics.

\textbf{Disciplinary integration:}
\begin{itemize}
    \item Physics: Experimental quantum systems and control.
    \item Mathematics: Control theory and dynamical systems.
    \item Informatics: Optimization and control systems.
    \item Quantum Chemistry: Interface materials and decoherence modeling.
\end{itemize}

\textbf{Local relevance:} UiO, SINTEF and other?

\textbf{Career paths:} Experimental physicist, quantum engineer, cryogenic systems designer.

\end{frame}


\begin{frame}[plain,fragile]
\frametitle{Study Direction 6: Quantum Machine Learning}
\textbf{Focus:} Quantum algorithms for artificial intelligence and data-driven discovery.

\textbf{Content:} Quantum neural networks, quantum generative models, hybrid quantum-classical algorithms.

\textbf{Disciplinary integration:}
\begin{itemize}
    \item Physics: Variational algorithms and circuit design.
    \item Mathematics: Statistics and optimization.
    \item Informatics: Machine learning and AI frameworks.
    \item Quantum Chemistry: Data-driven molecular prediction.
\end{itemize}

\textbf{Local relevance:} SimulaMet, NBIM, DNB?

\textbf{Career paths:} Quantum ML researcher, AI engineer, data scientist in R\&D.

\end{frame}


\begin{frame}[plain,fragile]
\frametitle{Summary Table of Study Directions}
\footnotesize{
\begin{center}
\begin{tabular}{p{3.5cm}p{3cm}p{3cm}p{3cm}p{3cm}}
\toprule
\textbf{Study Direction} & \textbf{Focus} & \textbf{Key Disciplines} & \textbf{Local Relevance} & \textbf{Career Paths} \\
\midrule
Quantum Computing \& Algorithms & Algorithms and theory & Physics, Math, Informatics, Chemistry & UiO, Simula & Research, Software, Academia \\
Quantum Materials \& Simulations & Quantum materials and modeling & Physics, Math, Chemistry, Informatics & SMN, Hylleraas & Simulation scientist, R\&D \\
Quantum Information \& Communication & Secure communication, entanglement & Physics, Math, Informatics & UiO, SINTEF & Cryptography, Networking \\
Quantum Sensing \& Metrology & Precision measurement & Physics, Math, Chemistry, Informatics & UiO, SINTEF & Sensor engineer, Researcher \\
Quantum Control \& Engineering & Hardware and control systems & Physics, Math, Informatics & SINTEF & Hardware engineer, Physicist \\
Quantum Machine Learning & AI and hybrid quantum computing & Physics, Math, Informatics & Simula & Data scientist, Researcher \\
\bottomrule
\end{tabular}
\end{center}
}
\end{frame}

\begin{frame}[plain,fragile]
\frametitle{Or fewer study directions}
\begin{itemize}
\item Quantum information and communication
\item Quantum Computing
\item Quantum Sensing and Metrology
\item Quantum machine learning
\end{itemize}
or iterations thereof.
\end{frame}



\begin{frame}[plain,fragile]
\frametitle{Existing and new courses that can be used, all 10 ECTS}
\begin{itemize}
    \item FYS3415/4415 Quantum Computing Fundamentals
    \item MAT3420 Quantum Computing Fundamentals
    \item MAT4430 Quantum information theory
    \item FYS4480 Many-body physics
    \item FYS5419 Quantum Computing and Quantum Machine Learning
    \item FYS5429 Advanced Machine Learning
    \item FYS-MENA4111 Quantum Mechanical Modelling of Nanomaterials
    \item FYS4411: Computational Physics 2, Monte Carlo methods
    \item FYS4170: Quantum field theory
    \item FYS4110: Modern quantum mechanics
\end{itemize}

\end{frame}


\begin{frame}[plain,fragile]
\frametitle{To be developed?}
    \item Quantum Mechanics for Quantum Technologies
    \item Mathematics for Quantum Science 
    \item Programming for Quantum Simulations
    \item Quantum Information Theory and Algorithms
    \item Quantum Hardware
    \item Quantum Sensors and Metrology
    \item Cryptography and cybersecurity
\begin{itemize}

\end{itemize}

\end{frame}



\end{document}

