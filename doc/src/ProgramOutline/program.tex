%%
%% Automatically generated file from DocOnce source
%% (https://github.com/doconce/doconce/)
%% doconce format latex program.do.txt --minted_latex_style=trac --latex_admon=paragraph --no_mako
%%


%-------------------- begin preamble ----------------------

\documentclass[%
oneside,                 % oneside: electronic viewing, twoside: printing
final,                   % draft: marks overfull hboxes, figures with paths
10pt]{article}

\listfiles               %  print all files needed to compile this document

\usepackage{relsize,makeidx,color,setspace,amsmath,amsfonts,amssymb}
\usepackage[table]{xcolor}
\usepackage{bm,ltablex,microtype}

\usepackage[pdftex]{graphicx}

\usepackage[T1]{fontenc}
%\usepackage[latin1]{inputenc}
\usepackage{ucs}
\usepackage[utf8x]{inputenc}

\usepackage{lmodern}         % Latin Modern fonts derived from Computer Modern

% Hyperlinks in PDF:
\definecolor{linkcolor}{rgb}{0,0,0.4}
\usepackage{hyperref}
\hypersetup{
    breaklinks=true,
    colorlinks=true,
    linkcolor=linkcolor,
    urlcolor=linkcolor,
    citecolor=black,
    filecolor=black,
    %filecolor=blue,
    pdfmenubar=true,
    pdftoolbar=true,
    bookmarksdepth=3   % Uncomment (and tweak) for PDF bookmarks with more levels than the TOC
    }
%\hyperbaseurl{}   % hyperlinks are relative to this root

\setcounter{tocdepth}{2}  % levels in table of contents

\usepackage[framemethod=TikZ]{mdframed}

% --- begin definitions of admonition environments ---

% --- end of definitions of admonition environments ---

% prevent orhpans and widows
\clubpenalty = 10000
\widowpenalty = 10000

% --- end of standard preamble for documents ---


% insert custom LaTeX commands...

\raggedbottom
\makeindex
\usepackage[totoc]{idxlayout}   % for index in the toc
\usepackage[nottoc]{tocbibind}  % for references/bibliography in the toc

%-------------------- end preamble ----------------------

\begin{document}

% matching end for #ifdef PREAMBLE

\newcommand{\exercisesection}[1]{\subsection*{#1}}


% ------------------- main content ----------------------



% ----------------- title -------------------------

\thispagestyle{empty}

\begin{center}
{\LARGE\bf
\begin{spacing}{1.25}
Including Quantum Technologies and AI/ML in educational programs at the Department of Physics, UiO
\end{spacing}
}
\end{center}

% ----------------- author(s) -------------------------

\begin{center}
{\bf Marianne Etzelm\"uller Bathen, Morten Hjorth-Jensen, and Lasse Vines}
\end{center}

    \begin{center}
% List of all institutions:
\centerline{{\small Department of Physics, UiO}}
\end{center}
    
% ----------------- end author(s) -------------------------


% --- begin date ---
\begin{center}
Planned start: Fall 2024 (?)
\end{center}
% --- end date ---

\vspace{1cm}


% !split
\subsection*{Establishing new study directions in the Physics and Astronomy BSc program and Master of Science in Physics}


% --- begin paragraph admon ---
\paragraph{}
We would like to propose
\begin{enumerate}
\item A new study direction under the Physics and Astronomy (PA) BSc program called
\begin{itemize}

  \item \textbf{Quantum technologies and AI/ML} (name to be discussed)

  \item Planned start fall 2024 for the new study direction

\end{itemize}

\noindent
\item At a later stage, a possible name change of the PA BSc program to for example
\begin{itemize}

  \item Physics, Astronomy and Quantum Technologies

\end{itemize}

\noindent
\item Similarly, the Physics MSc program changes name to
\begin{itemize}

  \item Physics and Quantum Technologies

  \item With a study direction in Quantum Technologies/Science

  \item Computational Physics and AI/ML in the Physical Sciences
\end{itemize}

\noindent
\end{enumerate}

\noindent
% --- end paragraph admon ---




% --- begin paragraph admon ---
\paragraph{Possible collaboration with:}

\begin{itemize}
 \item Department of Chemistry

 \item Department of Informatics

 \item Department of Mathematics
\end{itemize}

\noindent
The program is  administrated by the Department of Physics.
% --- end paragraph admon ---



% !split
\subsection*{Strategic importance}


% --- begin paragraph admon ---
\paragraph{}
Computational physics, computational science and data science play a
central role in scientific investigations and are central to
innovation in most domains of our lives. These fields underpin the
majority of today's technological, economic and societal feats. We
have entered an era in which huge amounts of data offer enormous
opportunities, but only to those who are able to harness them. The 3rd
industrial revolution will alter significantly the demands on the
workforce. In particular, the developments taking place in quantum
technologies and quantum information systems (QIS) together with
artificial intelligence (AI) and machine learning (ML) are expected to
play a significant role in technology developments and innovations,
and for fundamental discoveries in physics.
% --- end paragraph admon ---



% !split
\subsection*{AI and machine learning}


% --- begin paragraph admon ---
\paragraph{}
Artificial
intelligence is built upon integrated machine learning algorithms,
which in turn are fundamentally rooted in optimization and statistical
learning.
% --- end paragraph admon ---



% !split
\subsection*{AI and ML in Physics}


% --- begin paragraph admon ---
\paragraph{}
Artificial intelligence (AI) and Machine learning (ML)  techniques
have in the last years gained considerable traction in
scientific discovery. In particular, applications and techniques for
so-called \textbf{fast ML}, that is high-performance ML methods applied
to real time experimental data processing, hold great promise for
enhancing scientific discoveries in many different disciplines.
These developments cover a broad mix of rapidly
evolving fields, from the development of ML techniques to computer and
hardware architectures.
% --- end paragraph admon ---



% !split
\subsection*{Physics based Machine Learning}

An important and emerging field is what has been dubbed as scientific ML, see the article by Deiana et al \href{{https://arxiv.org/abs/2110.13041}}{Applications and Techniques for Fast Machine Learning in Science, arXiv:2110.13041}


% --- begin paragraph admon ---
\paragraph{}
The authors discuss applications and techniques for fast machine
learning (ML) in science -- the concept of integrating power ML
methods into the real-time experimental data processing loop to
accelerate scientific discovery. The report covers three main areas

\begin{enumerate}
\item applications for fast ML across a number of scientific domains;

\item techniques for training and implementing performant and resource-efficient ML algorithms;

\item and computing architectures, platforms, and technologies for deploying these algorithms.
\end{enumerate}

\noindent
% --- end paragraph admon ---



% !split
\subsection*{Many new research directions}


% --- begin paragraph admon ---
\paragraph{}
For our research in for example particle and
nuclear physics, fields which cover a huge range of energy and length scales,
spanning from our smallest constituents to the physics of dense
astronomical objects like supernovae and neutron stars, AI and ML
techniques offer possibilities for new discoveries and deeper insights
about the physics of atomic nuclei, elementary particles and dense
matter. Similarly, ML algorithms are widely applied in condensed
matter physics, materials science and nanotechnology, in molecular dynamics simulations of complex
systems in neuroscience and in many other fields in natural science.
% --- end paragraph admon ---




% --- begin paragraph admon ---
\paragraph{Examples of applications in subatomic physics.}
\begin{itemize}
\item \textbf{Artificial Intelligence and Machine Learning in Nuclear Physics}, Amber Boehnlein et al., \href{{https://arxiv.org/abs/2112.02309}}{arXiv:2112.02309} and Reviews of Modern Physics, 2022, in press

\item \textbf{Predicting Solid State Material Platforms for Quantum Technologies}, Hebnes et al,. \href{{https://arxiv.org/abs/2203.16203}}{arXiv:2203.16203}

\item \href{{https://arxiv.org/abs/1803.08823}}{Mehta et al.} and \href{{https://www.sciencedirect.com/science/article/pii/S0370157319300766?via%3Dihub}}{Physics Reports (2019)}.

\item \href{{https://link.aps.org/doi/10.1103/RevModPhys.91.045002}}{Machine Learning and the Physical Sciences by Carleo et al}

\item \href{{https://pdg.lbl.gov/2021/reviews/rpp2021-rev-machine-learning.pdf}}{Particle Data Group summary on ML methods}
\end{itemize}

\noindent
% --- end paragraph admon ---



% !split
\subsection*{Quantum Information Technologies (QIT)}


% --- begin paragraph admon ---
\paragraph{}
Recent developments in quantum information systems
and technologies offer the possibility to address some of the most
challenging large-scale problems, whether they are represented by
complicated interacting quantum mechanical systems or classical
systems.  Originally proposed by Feynman, the efficient simulation of
for example quantum systems by other, more controllable quantum
systems formed the basis for modern constructions of quantum
computations.  Many algorithmic and theoretical advances have followed
since the initial work in this area and with recent developments in
quantum computing hardware there is an additional drive to identify
early practical problems on which these devices might demonstrate an
advantage.
% --- end paragraph admon ---



% !split
\subsection*{More on QIT}


% --- begin paragraph admon ---
\paragraph{}
In addition to theoretical activities conducted at the
Department of Physics (mainly at the Center for Computing in Science
Education (CCSE) and the condensed matter group and other groups), there is a growing
interest to study candidate systems for making quantum hardware. In
particular, so-called point defects in semiconductors are pursued by
experimenters at the center for Materials Science.  With this broad
list of activities at the department of physics, there is a huge
potential to prepare the ground for educating physicists with the
theoretical and experimental background needed for the 21st
century. There is also a great interest in candidates with such a
background, knowledge, skills and competences in industry and the
public sector.
% --- end paragraph admon ---



% !split
\subsection*{Why such a change?}


% --- begin paragraph admon ---
\paragraph{}
Establishing such educational directions will be
unique in Norway and has the potential to attract excellent students.
The popularity of the Computational Science and in particular the Computational Physics and Computational Materials Science study direction are clear indicators that these are fields with the potential to attract new students.
% --- end paragraph admon ---




% --- begin paragraph admon ---
\paragraph{}
Furthermore,
Oslo Metropolitan university  has recently acquired two quantum quantum computers, see \href{{https://kommunikasjon.ntb.no/pressemelding/oslomet-avduker-norges-forste-kvantedatamaskin?publisherId=15678779&releaseId=17917781}}{\nolinkurl{https://kommunikasjon.ntb.no/pressemelding/oslomet-avduker-norges-forste-kvantedatamaskin?publisherId=15678779&releaseId=17917781}} and is now establishing research and educational initiatives in quantum information systems. There are thus several interesting avenues for joint collaborations in quantum information systems and quantum technologies as well as developing joint educational programs.
% --- end paragraph admon ---



% !split
\subsection*{More on motivation}

Computational physics plays a central role in the above mentioned
developments.  Computations are simply indispensable.  At the
department of physics of the university of Oslo this is reflected in
the extremely popular study direction Computational Physics of the
master of science (MSc) program Computational Science. This program
has over the last two decades recruited many excellent students,
resulting in highly attractive candidates in academia and in industry
and the public sector. A large fraction of these students have
specialized either in artificial intelligence and machine learning
and/or in quantum information systems.  The large majority of the
these students have job offers at least one year before completing
their MSc theses. The program has also become one of the most
selective master programs at the University of Oslo, requiring a grade
average of 4.7 for entry in 2021. Furthermore, with recent advances in
quantum technologies, there is a strong potential for new developments
in the fields of nanotechnology and materials science, with the
possibilities to develop new experimental activities.

% !split
\subsection*{Rationale}


% --- begin paragraph admon ---
\paragraph{}
The rationale behind proposing such new study directions is:
\begin{enumerate}
\item To attract at an earlier stage new students with an explicit interest in QIS, QT and AI and ML in physics. 

\item To enhance the recruitment to fields in physics which are in high demand for students and candidates with an expertise in computations, QIS, QT, AI and ML. We expect high demands from both the private sector and the public sector for candidates with these competences, insights and skills.

\item Candidates with such a background will be of great importance for new scientific discoveries and technological innovations. At the department of physics of the university of Oslo there are several research directions whose scientific activities will benefit at large from candidates with such a background, spanning from fast ML for new discoveries to the development of QTs.   
\end{enumerate}

\noindent
% --- end paragraph admon ---



% !split
\subsection*{Structure of Program and Courses}

In developing such a program 
the Center for Computing in Science Education (CCSE) at the university of Oslo (UiO) could be the entity which provides the pedagogical research resources. It has the needed research experience
on how to design curricula so that students develop deep knowledge that is connected and useful.

% !split
\subsection*{Using recent advances in pedagogical research and defining potentially new research directions in education rsearch}

The reason why we believe the CCSE should be involved in planning this BSc program is that it has the necessary expertise to address several issues, such as


% --- begin paragraph admon ---
\paragraph{}
\begin{enumerate}
\item Lots of conceptual learning: superposition, entanglement, QIS and QT, etc. 

\item Connecting statistics and mathematics with ML methods

\item Linking ML algorithms with quantum ML.

\item Coding is indispensable. This is a central reason why the CCSE should be involved.

\item Experience with teamwork, project management, and communication are important and highly valued.

\item Experience with engagement with industry, public sector and priority to diversity through the Computational Science program and other activities at the CCSE.

\item Mentorship should begin the moment students enroll. The experiences with the Computational Science program developed at the CCSE will be of great value, as the activities of the KURT center.
\end{enumerate}

\noindent
% --- end paragraph admon ---



% !split
\subsection*{Societal needs}


% --- begin paragraph admon ---
\paragraph{}
The program aims at addressing future societal needs, such as the  needs for specialized candidates (Master of Science, PhDs, postdocs), but also the needs of  people with a broad overview of what is possible in  QIS and QT. There are  not enough potential employees in AI, ML, QIS and QT. There is  a clear supply gap.
% --- end paragraph admon ---




% --- begin paragraph admon ---
\paragraph{}
A BSc degree  with specialization  is thus a good place to start. Linking this with the Physics MSc  program and the Computational Science program and the study directions Computational physics and Computational materials science, will offer our various research fields top candidates as well as pointing to new research directions.
% --- end paragraph admon ---




% --- begin paragraph admon ---
\paragraph{}
And what about AI/ML and quantum technologies towards high-school?
% --- end paragraph admon ---



% !split
\subsection*{Paths in the BSc program}


% --- begin paragraph admon ---
\paragraph{}
The program could offer  three possible directions
\begin{enumerate}
\item Quantum information systems and quantum technologies

\item Artificial intelligence and machine learning in physics

\item Computational Physics
\end{enumerate}

\noindent
The students specialize in these directions in their last year of the BSc program.
% --- end paragraph admon ---



% !split
\subsection*{Structure and courses}

% --- begin paragraph admon ---
\paragraph{}
There are several existing courses which can be included in this program. There are also courses which need to be established. At the CCSE we have the research and educational expertise to establish two to three new courses in these directions. Most likely there are potential teachers from other groups.

A separate application for establishing these courses follows. The additional courses we propose are (these are suggestions and codes are tentative) listed here. Note that we propose these courses as cloned courses. We may consider extensions of these codes in order to offer PhD variants as well.

The list here is tentative. Two of the courses, FYS4446 and FYS4447 have been taught as special topics since 2018 and we have already a large body of material.
This list is a mere suggestion.

\begin{enumerate}
\item Classical and quantum laboratory, needs to be established, perhaps by researchers at the Center of Materials Science (LENS group), FYS2440

\item Quantum computing and software, needs to be established. This can be organized together with OsloMet and Simula Research lab.

\item Quantum hardware, needs to be established, this can be organized together with OsloMet and Simula Research lab. 

\item Quantum computing and quantum machine learning, FYS3446/FYS4446 (cloned course, already developed by CCSE)

\item Advanced machine learning and data analysis for the physical sciences, FYS3447/FYS4447 (already developed by CCSE)
\end{enumerate}

\noindent
The last three courses are elective ones for the last semester of study. Some of these courses can also be split into modules a 5 ECTS or 7.5 ECTS.
There are obviously other course alternatives. 
The first year is identical with the BSc program \textbf{Physics and Astronomy}.
% --- end paragraph admon ---



% !split
\subsection*{Structure and courses}

% --- begin paragraph admon ---
\paragraph{}
The table here is an example of a suggested path for a Master of Science project,
with course work the first year and thesis work the last year.


\begin{quote}
\begin{tabular}{llll}
\hline
\multicolumn{1}{l}{  } & \multicolumn{1}{l}{ 10 ECTS } & \multicolumn{1}{l}{ 10 ECTS } & \multicolumn{1}{l}{ 10 ECTS } \\
\hline
6th semester & Elective/ExPhil & Elective/ML courses & FYS3XXX Quantum Computing, new                    \\
\hline
5th semester & FYS2160         & FYS3110             & FYS3XXX Quantum Materials, new                    \\
\hline
4th semester & FYS2130         & FYS2140             & FYS3150/FYS2150                                   \\
\hline
3rd semester & MAT1120         & FYS1120             & FYS1XXX Introduction to Quantum Technologies, new \\
\hline
2nd semester & MAT1110         & STK-FYS1100         & FYS1105                                           \\
\hline
1st semester & MAT1100         & IN1900              & FYS1100                                           \\
\hline
\end{tabular}
\end{quote}

\noindent
% --- end paragraph admon ---



% !split 
\subsection*{Description of Study directions}

The basic structure of the study directions could be

\begin{itemize}
\item Description of study directions with potential projects

\item Admission criteria

\item Learning outcomes

\item Program structure

\item Semester abroad

\item Career prospects

\item Teaching and examinations
\end{itemize}

\noindent
What follows are text proposals for these items.

% !split
\subsection*{Description of learning outcomes}

The power of the scientific method lies in identifying a given problem
as a special case of an abstract class of problems, identifying
general solution methods for this class of problems, and applying a
general method to the specific problem (applying means, in the case of
computing, calculations by pen and paper, symbolic computing, or
numerical computing by ready-made and/or self-written software). This
generic view on problems and methods is particularly important for
understanding how to apply available, generic software to solve a
particular problem.

Computing competence represents a central element
in scientific problem solving, from basic education and research to
essentially almost all advanced problems in modern
societies. Computing competence is simply central to further
progress. It enlarges the body of tools available to students and
scientists beyond classical tools and allows for a more generic
handling of problems. Focusing on algorithmic aspects results in
deeper insights about scientific problems.

The learning outcomes are subdivided in three general categories, knowledge, skills and general competence.

% !split
\subsection*{Study abroad and international collaborators}


% --- begin paragraph admon ---
\paragraph{}

Students at the University of Oslo may choose to take parts of
their degrees at a university abroad.

Students in this program have a number of interesting international
exchange possibilities. The involved researchers have extensive
collaborations with other researchers worldwide. These exchange
possibility range from top universities in the USA, Asia and Europe as
well as leading National Laboratories in the USA.
% --- end paragraph admon ---



% !split
\subsection*{Career prospects}


% --- begin paragraph admon ---
\paragraph{}
Candidates who are capable of modeling and understanding complicated
systems in natural science, are in short supply in society.  The
computational methods and approaches to scientific problems students learn
when working on their thesis projects are very similar to the methods
they will use in later stages of their careers.  To handle large
numerical projects demands structured thinking and good analytical
skills and a thorough understanding of the problems to be solved. This
knowledge makes the students unique on the job market.

Career opportunities are many, from research institutes, universities
and university colleges and a multitude of companies. Examples
include IBM, Hydro, Statoil, and Telenor.  The program gives an
excellent background for further studies.

The program has also a strong international element which allows students to
gain important experience from international collaborations in
science, with the opportunity to spend parts of the time spent on
thesis work at research institutions abroad.
% --- end paragraph admon ---



ETH
MSc in quantum engineering
https://master-qe.ethz.ch/
https://www.nature.com/articles/s41563-021-01080-6.epdf?sharing_token=hzaL6bfnmuTfmbxDqAhxkNRgN0jAjWel9jnR3ZoTv0MXCITNgQctnLQoGRWhf21yaRB4q5nw18vEzyc8SV_P-jJRufnLzILslhZSFq_PQE-o-160_paOACFYc2ZoEHSyLdsafCp69C0o6krGYLR8EFa_MOv5lg1Cto7SG46IQjY=

Aalto university
BSc and MSc in QT
https://www.aalto.fi/en/study-options/quantum-technology-bachelor-of-science-technology-master-of-science-technology
Qt specific courses
Introduction to quantum technology
Quantum materials
Quantum information
Quantum labs
https://www.aalto.fi/en/programmes/aalto-bachelors-programme-in-science-and-technology/curriculum-2022-2024#1-major-studies
Quantum circuits
BSc thesis
Photonics, microscopy, nanotechnology, materials physics, advanced quantum mech, quantum games, practical quantum computing and quantum machine learning are optional

University of Surrey
Physics with QT
BSc and MSc
https://www.surrey.ac.uk/undergraduate/physics-quantum-technologies#structure
Not much QT specific on BSc level

University of Copenhagen
MSc in quantum information science
https://studies.ku.dk/masters/quantum-information-science/
Three compulsory courses
Introduction to Quantum Information Science (UCPH)
Introduction to Quantum Computing (UCPH)
Applied Quantum Physics: Quantum Information Technology (DTU)

MIT
Paper: Building a Quantum Engineering Undergraduate Program
https://dspace.mit.edu/bitstream/handle/1721.1/143817/Building_a_Quantum_Engineering_Undergraduate_Program.pdf?sequence=2

West Chester and Uni Delaware
https://www.wcupa.edu/communications/newsroom/2023/03.07engineering.aspx
Combo BSc and MSc

Harvard
https://gsas.harvard.edu/program/quantum-science-and-engineering
Graduate, so MSc and PhD

Universities Chicago
https://chicagoquantum.org/education-and-training/undergraduate-and-graduate-education
Mest MSc og PhD

Cornell
https://quantum.cornell.edu/education/
San Jose state uni
https://www.sjsu.edu/quantum/
MSc in quantum tech

Kursinnhold 

Må lage nye kurs om vi skal ha studieretning i kvanteteknologi.

Nye kurs vi må lage 

Introduksjon til kvanteteknologier (3 semester) 
Grunnleggende kurs 
Krever kun linalg som forkunnskap 
Kan tas også av andre, e.g.~IT og kjemistudenter 
Marianne, Lasse, Johannes?
Innhold 
Bits 
Qubits
Kvanteteknologier 
Kvanteinfo
Litt om Algoritmer, grover og shor f eks 
Lab 
Måle transistor, sammenligne med qubits
Qiskit, måle qubits 
Kanskje se på enkel Grover Shor algoritmer  
Quantum materials (5 semester) 
Slags kompo av eksperimentalfysikk, faststoff og kvantehardware 
Må ha noe om strukturer, kan ha XRD lab 
Må ha noe om båndstruktur, kan ha UV VIS lab  
Snakke om bits, qubtis 
Gjøre litt fabrikasjon, eller spare det til senere?
Kvantealgoritmer (6 semester) 
Algoritmer og kvanteinformasjonsteori 
Johannes, Morten, Joakim? 
Har allerede et lignende kurs 
Blir som å lage et halvt nytt kurs 
Quantum materials 2? 
MSc kurs 
Resten av kondenserte 1 så man kan ta kondenserte 2 
Døpe om FysEd? 
Kombinere med nye MENA9510? 

Kurs 1: Introduksjon til kvanteteknologier 
(Introduction to quantum technologies) 
Innhold 
Motivasjon 
Basic kvantefysikk/ QT at a glance 
1st and 2nd quantum  
Kvantebits vs klassiske bits 
Materialer og praktiske kvanteplattformer 
Kvantesensorer 
Kvantekommunikasjon og kryptering 
Kvanteberegninger til slutt fordi linalg er parallelt  

Læringsmål 
Hovedmål: generell introduksjon til kvanteteknologi som gir oversikt over hele fagfeltet 
Forstå forskjellen mellom qubits og klassiske bits 

Ukesplan 
Motivasjon for kvanteteknologi og oppbygging av faget 
Popvit oversikt over hele kurset 
Oversikt over kurset 
Kvanteteknologier; inndeling i sensorer, kommunikasjon og computing 
Fra klassisk til moderne fysikk
Black-body radiation
Fotoelektrisk effect og fotoner
UV catastrophe 
Compton
De Broglie 
Grunnleggende konsepter i kvantefysikk 
Kvantisering av energinivåer 
Bølgefunksjon 
Partikkel i boks 
Fra enkeltatom til fast stoff  
First and second quantum revolution 
Halvleder, transistor, minne 
Fra det makroskopiske til det mikroskopiske 
Utnytte de mest eksotiske delene av kvantefysikken 
Tunnelering 
Entanglement 
superposition  
NMR 
Tunneleringsdioder 
Utnyttelse av superposisjon og entanglement; fra ide til virkelighet 
First and second quantum revolution 
Halvleder, transistor, minne 
Fra det makroskopiske til det mikroskopiske 
Utnytte de mest eksotiske delene av kvantefysikken 
Tunnelering 
Entanglement 
superposition  
Måling i kvanteteknologi 
NMR 
Tunneleringsdioder 
Utnyttelse av superposisjon og entanglement; fra ide til virkelighet 
Byggeklosser for klassisk og kvanteteknologi 
Bits
Qubits 
Krav og egenskaper til qubits 
Superposisjon og entanglement i praksis 
Lab; måle transistorer 
Lab; måle qubits i qiskit 
Materialplattformer for qubits 
Superledere 
Trapped ion 
Halvleder (quantum dot, point defects) 
Optisk 
Andre plattformer? Vapor phase greier 
Fabrikasjonsteknologier for kvanteteknologi (halvleder, superleder, trapped ion, optisk) 
Nanoteknologi 
Renrom 
Lasere og optikk 
Kvantesensorer
Definisjon og Konsept, meterologi 
Eksempler 
Atomic clock 
NMR 
LIGO 
NV i diamant i celler  
Kvantekommunikasjon 
Sikkerhet i kommunikasjon med klassik kryptering 
Måling i kvantefysikk/ measurement of the quantum state   
QKD 
Utfordringer: repeaters osv 
Kvantecomputing; Klassisk vs kvantecomputing 
Klassiske kretser og logic gates 
Quantum gates 
Adiabatisk kvantecomputing 
Kvantekretser
Lab 
Kvantecomputing; Klassisk vs kvantecomputing 
Klassiske kretser og logic gates 
Quantum gates 
Adiabatisk kvantecomputing 
Kvantekretser
Kvantecomputing; Kvantealgoritmer og quantum information 
Shor 
Grover
Teste lab, qiskit – Grover 
Hele informasjonssystemet 
Kvantecomputing; Status and future  
Noisy qubits 
QEC 
NISQ 
Quantum machine learning 
Wrap up 
Tillegg til forelesninger 
Undervisningsform: 
Forelesning, 2 timer per uke?  
Diskusjonspoppgaver i plenum 
Menti 

Ukesoppgaver: 
Gruppeoppgaver for diskusjon hver uke 
Regneoppgaver noen uker – oblig?? 

Lab:
Eksp 1: Målinger av bits; dioder og transistorer 
MiNa 
Probe station 
Eksp 2: Noe material-relatert med høy throughput av studenter 
Teori 1: qiskit, måling av qubits
Teori 2: qiskit, sette opp enkle kvantekretser, Grover 

Vurdering: 
Midtveisvurdering 
Eksamen
Multiple choice? 
Prosjekt er urealistisk for mange studenter 

Pensumlitteratur  
Lage kompendium? 

Kurs 2: Quantum materials 

Ressurser fra lignende kurs
TU Delft 
https://ocw.tudelft.nl/course-lectures/1-3-1-quantum-materials/ 
Quantum materials provide the environment where qubits, the elemental unit of quantum information processing, are defined and live.
Precision in material uniformity, chemical composition and electrical properties are crucial for the requirements of having both many qubits and long decoherence times.
Chemical Vapour Deposition is an industrial process that uses high purity gases to make high quality materials, with desired physical and electronic properties.
Transmission electron microscopy is a process to inspect the fabricated heterostructures with high resolution.
Temperature, electric fields and magnetic fields are useful parameters to determine properties of quantum materials such as mobility and electron density.

Uppsala university 
https://www.uu.se/en/study/course?query=1FA654 
MSc level course 
Learning outcomes
On completion of the course, the student should be able to:
present and apply theoretical condensed matter models and evaluate their applicability under various conditions
classify condensed matter based on their electronic structure
apply Dirac formalism to perform quantum mechanical analysis of relevant systems
carry out and explain experiments in condensed matter physics 
Reading list: 
Sakurai, J. J.; Napolitano, Jim., Modern Quantum Mechanics, Third edition., Cambridge, Cambridge University Press, 2021Compulsory*
Kittel, Charles; McEuen, Paul, Introduction to solid state physics, 8. ed., Hoboken, N.J., Wiley, c2005Compulsory*
Content: 
Quantum mechanical formulation of Bloch functions in a periodic crystal. Single particle models for electrons. Descriptions of electrons and quasiparticles with the Schrödinger and Dirac equations. Classification of materials based on their electronic structure. Introductory laboratory experiments in connection to solid state physics. Introduction to emerging phenomena with Dirac materials, conventional and unconventional superconductors, spintronic and topological materials. Short introduction to second quantization.

KTH 
https://www.kth.se/student/kurser/kurs/SK2904?l=en 
This course addresses future quantum materials, where control of the electron spin opens possibilities for a new technological era, the "quantum age". Their unique material properties will be described together with a physical description of how such properties occur, and also how these materials have the potential to generate new technological devices and applications for a sustainable society. Furthermore, the course will describe the most important experimental characterization methods used to understand these complex materials down to a subatomic level.
Intended learning outcomes
After the course, students should be able to:
Describe the characteristics of different quantum materials and explain the physical background to these unique characteristics.
Explain how these materials can be used in future technical applications. 
analyze how new quantum materials can affect development towards a sustainable society. 
Assess which experimental methods are best suited to characterize the properties of quantum materials.
Examination • LAB1 - Laboratory work, 2.0 credits, grading scale: P, F • PRO1 - Poject work, 5.5 credits, grading scale: A, B, C, D, E, FX, F
Dear participating students of SK2904 "Quantum Materials". The course is given in a mixed format where I will first give a set of general introductory lectures on quantum materials as well as advanced experimental techniques that are utilized to study such materials. There are then a set of topics (see list below) where each student will select one topic to focus on and prepare a presentation about. 
Topological Insulators vs.~Weyl semimetals
Two-Dimensional Materials
Skyrmions
Quantum Magnets
Nanostructured semiconductors
NV Centers in Diamond
Ferroelectrics {\&} Multiferroics
High-Temperature Superconductors
Heavy Fermions
Manganite and Colossal Magnetoresistance
Ultra-Cold Atoms
Aalto University 
https://mycourses.aalto.fi/course/view.php?id=33560 

Using the links below you should be able to read the course book, Young and Freedman, University Physics (14th Edition), Pearson. 
Link to the book via Aalto university library https://primo.aalto.fi/permalink/358AALTO_INST/1rd34t9/alma997819554406526
 Topics 
• Wave motion: basic quantities, wave equation, wave velocity, energy, reflection and transmission, waves in 3 D, Snell’s law, interference, diffraction, dispersion, group velocity 
• Particle properties of light: photoelectric effect, X-ray generation and diffraction, Compton scattering, pair production, wave-particle duality, uncertainty relation 
• Wave properties of particles: Bohr model of hydrogen atom, wave nature of electron 
• Quantum mechanics of single particle: Schr¨odinger equation, potential wells, quantum tunneling, measurement, 3 D deep potential well, hydrogen atom, angular momentum, hydrogen probability distributions, hydrogen energy levels, electron spin 
• Many-particle systems: many-body problem, atomic structure and the periodic table, Pauli principle as symmetry, chemical bond, molecular rotations and vibrations 
• Solid state: energy bands, conductors and insulators, free electron model of metal, superconductivity, insulators, semiconductors 

Learning Outcomes (from Sisu): After the course, the student 
1. knows the basics of wave propagation 
2. can explain the formation of intensity distribution in interference and diffraction patterns 
3. knows the Heisenberg’s uncertainty principle and the wave-particle dualism in quantum mechanics
4. can solve simple quantum mechanical systems using Schr¨odinger’s equation 
5. can identify properties of atoms and condensed matter based on quantum mechanics 

Content (from Sisu): Interference, diffraction, dispersion, group- and phase velocity. Heisenberg’s uncertainty principle, wave-particle dualism. Schr¨odinger’s equation, statistical nature of quantum mechanics. Quantum numbers, Pauli exclusion principle, spin and periodic table of elements

https://mycourses.aalto.fi/pluginfile.php/1746418/mod_resource/content/2/mqmat.pdf 
Quantum Materials BSc course notes!! 

Innhold
Slags kombinasjon av faststoff, eksperimentalfysikk og quantum hardware
Krystallstruktur 
Båndstruktur 
Halvlederfysikk (enkel) 
Superledere 

Læringsmål 

Ukesplan 
Part 1: Condensed matter physics 
Introduksjon 
Crystal bonding
Lattices  
Reciprocal space 
Overview of course 
Crystals 
Bragg diffraction 
Brillouin zones 
XRD/TEM lab 
Phonons 
Vibration in atomic chains 
Dispersion relation 
Periodic boundary conditions, Born von Karman, DOS 
Phonons and heat 
Electrons in the wild 
Free electron gas 
DOS i 1D, 2D, 3D 
Transport properties of electrons 
Electrons in the solid state  
Origin of band gap 
Bloch functions 
Kronig penny model
Effective mass model  

Part 2: Quantum materials 
Trapped ion 
Bruk som tankeeksperiment
Manipulating single atoms 
Applications for QT, memory and computing maybe
Metals – superconductors  - 1 
Metals og Fermi surfaces 
BCS teori 
Meissner effect og energy gap 
Type 1 og type 2 
Superconductors – 2 
Josephson junctions 
SQUID 
Qubits 
Applications!
Magnetic field sensing 
Quantum computing
Construction of a quantum computer 
LAB 
Semiconductors – 1 
Band gap
Energy bands 
Doping 
Semiconductors – 2  
Pn junction quickly 
Heterojunctions in 1, 2 and 3 dimensions 
Surfaces, interfaces, strain 
Point defects 
Single photon emitters 
Electrical lab, DLTS 
Application of structures in semicondcutors for QT 
Quantum wells
Quantum dots
Point defects 
P in Si, Kane quantum computer 
Applications 
Sensing
Communication
Computing 
Lab
Application of structures in semiconductors for QT 
Quantum wells
Quantum dots
Point defects 
P in Si, Kane quantum computer 
Applications 
Sensing
Communication
Computing 
Lab
Emission of light from quantum materials
Excitons 
Defects
QDs
PL-lab  
Manipulation of photons for QT 
Squeezed light
LIGO 
SPDC 
Entanglement 
Applications for QT, communication 
Lab  
Application and wrap up, next coursed 

Tillegg til forelesninger 

Lab 
XRD for krystallstruktur 
Superlederlab 
Halvleder 
UV-VIS for båndstruktur 
Elektrisk lab 
CV, IV for enkle devices 
DLTS 
PL lab 
Helst en HBT type lab/ squeezed light  

DLTS for defekter?? 
PL?? 
Ideelt sett klarer vi hBT, HOM 
Renrom og nanofabrikasjon blir kanskje det mer avanserte kurset? 

Pensumlitteratur 


% ------------------- end of main content ---------------

\end{document}

