\documentclass[oneside,final,10pt]{article}

\listfiles               %  print all files needed to compile this document

\usepackage{relsize,makeidx,color,setspace,amsmath,amsfonts,amssymb}
\usepackage[table]{xcolor}
\usepackage{bm,ltablex,microtype}
\usepackage[pdftex]{graphicx}

\usepackage[T1]{fontenc}
%\usepackage[latin1]{inputenc}
\usepackage{ucs}
\usepackage[utf8x]{inputenc}

\usepackage{lmodern}         % Latin Modern fonts derived from Computer Modern

% Hyperlinks in PDF:
\definecolor{linkcolor}{rgb}{0,0,0.4}
\usepackage{hyperref}
\hypersetup{
    breaklinks=true,
    colorlinks=true,
    linkcolor=linkcolor,
    urlcolor=linkcolor,
    citecolor=black,
    filecolor=black,
    %filecolor=blue,
    pdfmenubar=true,
    pdftoolbar=true,
    bookmarksdepth=3   % Uncomment (and tweak) for PDF bookmarks with more levels than the TOC
    }

\setcounter{tocdepth}{2}  % levels in table of contents

\setcounter{topnumber}{2}
\setcounter{bottomnumber}{2}
\setcounter{totalnumber}{4}
\renewcommand{\topfraction}{0.95}
\renewcommand{\bottomfraction}{0.95}
\renewcommand{\textfraction}{0}
\renewcommand{\floatpagefraction}{0.75}
\clubpenalty = 10000
\widowpenalty = 10000

\raggedbottom
\makeindex
\usepackage[totoc]{idxlayout}   % for index in the toc
\usepackage[nottoc]{tocbibind}  % for references/bibliography in the toc

\begin{document}

\thispagestyle{empty}

\begin{center}
{\LARGE\bf
\begin{spacing}{1.25}
CompPhysQuantum: Computational Physics and Quantum Technologies  at the University of Oslo
\end{spacing}
}
\end{center}



\vspace{1cm}


\section*{Executive Summary}

We propose to establish, with startup Fall semester 2023, a new bachelor/undergraduate program in Computational Physics and Quantum Technologies  at the University of Oslo. 

\section*{Introduction and Background}

Computational Physics, Computational Science  and Data Science play a central role in scientific investigations and are central to innovation in most domains of our lives. These fields underpin the majority of today's technological, economic and societal feats. We have entered an era in which huge amounts of data offer enormous opportunities, but only to those who are able to harness them. The 3rd Industrial Revolution will alter significantly the demands on the workforce. In particular, the developments taking place in quantum technologies and quantum information systems together with artificial intelligence and Machine Learning are expected to play a significant role in technology developments and innovations and for potential discoveries in Physics.
Computational Physics plays a central role in these developments. At the Department of Physics of the University of Oslo this is reflected in the extremely popular study direction Computational Physics of the master of Science program Computational Science. 




\section*{Bachelor of Science program in Computational Physics and Quantum Technologies}









\paragraph{Mathematics}
Here we can think of the topics and themes covered by the central  Math courses we have presently.
30 ECTS in total
\begin{enumerate}
\color{red}
    \item \textcolor{red}{MAT1100}, 1st year
    \item \textcolor{red}{MAT1110}, 1st year
    \item \textcolor{red}{MAT1120}, 2nd year
\end{enumerate}

\paragraph{Programming and system development}
Here we can think of the topics and themes covered by the existing central programming courses.
20 ECTS plus 10 elective ECTS (also smaller 5 ECTS modules)
\begin{enumerate}
    \item \textcolor{red}{IN1900} 1st year
    \item \textcolor{red}{IN1910}
    \item IN3200, High-performance Computing, elective
\end{enumerate}

\paragraph{Computational Science}
20 ECTS + 10 ECTS elective (also smaller 5 ECTS modules)
\begin{enumerate}
    \item \textcolor{red}{Numerical Methods I} Could be revised version of MAT-INF1100, first year
    \item \textcolor{red}{Numerical Methods II}, second year, could be MAT3110
    \item Numerical Methods III, elective,  third year, FYS3150, Computational Physics, third year
\end{enumerate}

\paragraph{Probability and Statistics}
20 ECTS + 10 ECTS elective (also smaller 5 ECTS modules). Here we can think of courses tailored to specific disciplines as well. 

 In order to understand methods within data science (or perhaps here restricted to data analysis), uncertainty, random variables and probability are important concepts that needs to be learning. For most students these are new concepts that they use time to learn properly. 
\begin{enumerate}
    \item \textcolor{red}{Introduction to probability and statistics}, first year, preferable a revised version of STK1100.
    
    An introductory course (10 ECTS) should introduce probability theory (including conditional probabilities and Bayes theorem, some common distributions, expectations, variances, covariances), statisitcal inference (the likelihood function, parametric modelling, ML estimation, bias in estimation, confidence intervals Bootrapping, some simple models). In order to make the course attractive for several bachelor programs, the course should include good examples/case studies from substance disciplines.
    \item \textcolor{red}{Statistical learning and machine learning}, 2nd year, preferable STK2100
    
    A second course (10 ECTS) should focus on statistical/machine learning methods, starting with linear models (going a bit in depth on this) and then demonstrating \emph{some} advanced methods (e.g nearest neighbor, neural network). Possibly also examples of unsupervised learning/data mining.

    \item Data Analysis, AI and Statistics III, third year
    
    Possible courses here are STK3100 - Generalized linear models,  
IN3050 – Introduction to Artificial Intelligence and Machine Learning, FYS-STK3155 – Applied Data Analysis and Machine Learning 
\end{enumerate}




\end{document}

