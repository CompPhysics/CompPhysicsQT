\documentclass[aps,rmp,reprint,amsmath,amssymb,graphicx,longbibliography]{revtex4-1}

\usepackage{bm}
\usepackage{graphicx}
\usepackage{epstopdf}
\usepackage{wrapfig}
\usepackage{array} 
\usepackage{listings}
\usepackage[para,online,flushleft]{threeparttablex}
\usepackage{booktabs,dcolumn}
\usepackage{color}

 

\usepackage{textpos}
\usepackage{booktabs}
\usepackage{multirow,bigdelim}
\usepackage{float}

\usepackage{upgreek} %upalpha in Saxena2021 Reference

\usepackage[utf8]{inputenc}
\usepackage{hyperref}
\hypersetup{breaklinks=true,colorlinks=true,linkcolor=blue,citecolor=blue,filecolor=magenta,urlcolor=blue}

\usepackage{xcolor}
\newcommand{\contrib}[1]{\textcolor{red}{#1}}
\newcommand{\comment}[1]{\textcolor{blue}{#1}}
\newcommand{\WN}[1]{{\color{red} #1}}

\makeatletter
\def\@bibdataout@aps{%
\immediate\write\@bibdataout{%
@CONTROL{%
apsrev41Control%
\longbibliography@sw{%
    ,author="08",editor="1",pages="1",title="0",year="1"%
    }{%
    ,author="08",editor="1",pages="1",title="",year="1"%
    }%
  }%
}%
\if@filesw \immediate \write \@auxout {\string \citation {apsrev41Control}}\fi
}
\makeatother

\begin{document}

\title{\underline{Computational Physics and Quantum Technologies}; a new Bachelor of Science program  at the Department of Physics, University of Oslo}

\author{Morten Hjorth-Jensen and Anders Malthe-Sørenssen}
\affiliation{Department of Physics and Center for Computing in Science Education, University of Oslo}


\begin{abstract}
We propose to establish, with {\em startup Fall semester 2023}, a new bachelor/undergraduate program in {\bf Computational Physics and Quantum Technologies, (CompPhysQuantum)}  at the University of Oslo, hosted by the Department of Physics. 
\end{abstract}
\maketitle
\section{Introduction}

Computational Physics, Computational Science  and Data Science play a central role in scientific investigations and are central to innovation in most domains of our lives. These fields underpin the majority of today's technological, economic and societal feats. We have entered an era in which huge amounts of data offer enormous opportunities, but only to those who are able to harness them. The 3rd Industrial Revolution will alter significantly the demands on the workforce. In particular, the developments taking place in quantum technologies and quantum information systems together with artificial intelligence and Machine Learning are expected to play a significant role in technology developments and innovations and for potential discoveries in Physics.

Artificial intelligence and machine learning techniques\footnote{Artificial intelligence is built upon integrated machine learning algorithms, which in turn are fundamentally rooted in optimization and statistical learning.} have in the last years gained considerable traction in scientific discovery. In particular, applications and techniques for so-called {\em fast} ML, that is high-performance ML methods applied to real time experimental data processing, hold great promise for enhancing scientific discoveries in many different disciplines\cite{deiana2021}. 
These developments cover a broad mix of rapidly evolving  fields, from the development of ML techniques to computer and hardware architectures. For our research in for example particle and nuclear physics, which cover a huge range of energy and length scales, spanning from our smallest constituents to the physics of dense astronomical objects like supernovae and neutron stars, AI and ML techniques offer possibilities for new discoveries and deeper insights about the physics of atomic nuclei, elementary particles and dense matter. Similarly, machine leaning algorithms 



Artificial intelligence (and Machine learning)  is an extremely rich field, in spite of its young age. The
increases we have seen during the last three decades in computational
capabilities have been followed by developments of methods and
techniques for analyzing and handling large date sets, relying heavily
on statistics, computer science and mathematics.  The field is rather
new and developing rapidly. \cite{deiana2021}

\section{Background and Motivation}

Computational Physics plays a central role in the above mentioned   developments. At the Department of Physics of the University of Oslo this is reflected in the extremely popular study direction Computational Physics of the master of Science program Computational Science. This program has over the last two decades recruited many excellent students, resulting in highly attractive candidates in academia and in  the private and public sectors. Many of the these students have job offers at least one year before completing their Master of Science theses.

The rationale behind proposing a new Bachelor of Science program in Computational Physics and Quantum Technologies is:
\begin{enumerate}
    \item To attract at an earlier stage new students with an explicit interest in Quantum Technologies and Artificial Intelligence and Machine Learning in Physics. 
    \item To enhance the recruitment to fields in Physics which are in high demand for students and candidates with an expertise in computations, quantum information systems and artificial intelligence and machine learning. We expect high demands from both the private and the public sectors for candidates with these competences, insights and skills.
    \item Candidates with such a background will be of great importance for new scientific discoveries and for further technological discoveries. At the Department of Physics of the University of Oslo there are several research directions whose scientific activities will rely heavily on candidates with such a background, spanning from fast Machine Learning for new discoveries to the development of quantum technologies.   
\end{enumerate}


This will strengthen 



\section{Structure of Program and Courses}

!split
===== Quantum Engineering =====

!bblock Quantum Computing requirements
o be scalable
o have qubits that can be entangled
o have reliable initializations protocols to a standard state
o have a set of universal quantum gates to control the quantum evolution
o have a coherence time much longer than the gate operation time
o have a reliable read-out mechanism for measuring the qubit states
o and many more 
!eblock

!split
===== Candidate systems =====
o Superconducting Josephon junctions
o Single photons
o "Trapped ions and atoms":"https://www.insidequantumtechnology.com/news-archive/ionq-is-first-quantum-startup-to-go-public-will-it-be-first-to-deliver-profits/"
o Nuclear Magnetic Resonance
o _Quantum dots, expt at MSU_
o _Point Defects in semiconductors, experiments at UiO, center for Materials Science_
o more


!split
===== Courses, Prototype =====

The "Center for Computing in Science Education at UiO":"https://www.mn.uio.no/ccse/english/" could be the entity which provides the pedagogical resourses. It has research experience
on how do we design curricula so that students develop deep knowledge that is connected and useful.

!bblock Topics  in a Bachelor of Science/Master of Science
o Information Systems 
o From Classical Information theory to Quantum Information theory
o Classical vs. Quantum Logic
o Classical and Quantum Laboratory 
o Discipline-Based Quantum Mechanics 
o Quantum Software
o Quantum Hardware
o more
!eblock

!split
===== Important Issues to think of =====
o Lots of conceptual learning: superposition, entanglement, QIS applications, etc.
o Coding is indispensable. That is why this should be a part of a CS/DS program
o Teamwork, project management, and communication are important and highly valued
o Engagement with industry: guest lectures, virtual tours, co-ops, and/or internships.
o Diversity needs to be a priority
o Mentorship should begin the moment students enroll.

!split
===== Observations =====

o Students do not really know what QIS is. This may be important when advertising
o There is conflation of “Quantum Information Science” with “Quantum computing”.
o Students perceive that a graduate degree is necessary to work in QIS. A BSc will help.

!split
===== Future Needs/Problems (US observations mostly but transfer most likely to Europe as well) =====

o There are already (USA) great needs for specialized people (Ph. D. s, postdocs), but also needs of  people with a broad overview of what is possible in QIS.
o There are not enough potential employees in QIS (USA). It is a supply gap, not a skills gap.
o A BSc with specialization  is a good place to start
o It is tremendously important to get everyone speaking the same language. Facility with the vernacular of quantum mechanics is a big plus.
o There is a huge list of areas where technical expertise may be important. But employers are often more concerned with attributes like project management, working well in a team, interest in the field, and adaptability than in specific technical skills.



\begin{table}[hbtp]
 \caption{Basic Structure of the program}
    \centering
    \begin{tabular}{|c|c|c|c|} \hline
    First Semester & MAT1100 &  IN1900   & CS I  \\ \hline 
    Second Semester & MAT1110 &  DS I   & Elective \\ \hline
    Third Semester & MAT1120 &  IN1910   &  Elective\\ \hline
    Fourth Semester & CS II & DS II    & Elective \\ \hline       
    Fifth Semester & Elective/EX-PHIL & Elective     & Elective \\ \hline
    Sixth Semester & Elective/EX-PHIL & Elective    & Elective \\ \hline
   ECTS & 10 &  10   &  10  \\ \hline    
    \end{tabular}



\bibliography{References}


\end{document}

